\documentclass{beamer}

\usepackage[utf8x]{inputenc}
\usepackage{graphicx}
\usepackage{setspace}
\usepackage{tikz}
\usepackage{algorithm}
\usepackage{algpseudocode}
\usepackage{array}
\usepackage{amsmath}
\usepackage{pgfpages}

\usetikzlibrary{arrows,decorations.pathmorphing,backgrounds,positioning,fit,petri}

\usetheme{PaloAlto}
\usecolortheme{seahorse}
\usefonttheme{structurebold}

\setbeamertemplate{footline}
{
}
\setbeameroption{show notes on second screen}

\title{Docker for Fun and Profit}
\author{Luka Stojanović}
\institute{
  luka@magrathea.rs \\
  Seven Bridges Genomics
}
\date{Startit Tech Meetup \#3\\ 05. 12. 2013.}
\subject{Software engineering}

\begin{document}
  \begin{frame}
    \titlepage
  \end{frame}
  
  \begin{frame}
    \frametitle{Outline}
    \tableofcontents
  \end{frame}
  
  \section{What is Docker?}
  \begin{frame}
    \frametitle{What is Docker?}
    %\begin{block}{Title}
    %    Text
    %\end{block}
    
    \begin{block}{\url{http://www.docker.io}}
    	Docker is an open-source project to easily create lightweight, 
    	portable, self-sufficient containers from any application
    \end{block}
    \note{
		Now, questions
  	}
    
    %\begin{itemize}
    %    \item Item
    %\end{itemize}
    
  \end{frame}
  
  \begin{frame}
  	\frametitle{What does it mean?}
	\begin{itemize}  	
  		\item Open source
  	
	    \begin{itemize}
    		\item Code available on Github: \url{https://github.com/dotcloud/docker}
        	\item Written in Go
	        \item Actively developed, lots of community contributors
	    \end{itemize}
    
    \item Lightweight, portable, self-sufficient containers
    	\begin{itemize}
    		\item No virtual machine, runs on host hardware/kernel
			\item Isolated from host using kernel features such as control groups,
				process/network namespacing
	    \end{itemize}
    
    \item ...from any application
    	\begin{itemize}
    		\item Point of view that influences design
	    \end{itemize}
    \note{
		Code not of a supreme quality, but improving \\
		Go is easy to read and understand \\
		LXC is virtual-machine oriented \\
  	}
    \end{itemize}
    
  \end{frame}
  
  
  
  \begin{frame}
    \frametitle{What does it do?}
    \begin{itemize}
    	\item System daemon with an HTTP API
        \item Command line tool
	    \item Repository with pre-cooked images
	\end{itemize}
	\note{Show some cli usage, process on host}
  \end{frame}  
  
  \section{What's Inside the Box?}
  \begin{frame}
    \begin{itemize}
    	\item Linux containers (lxc)
        \item Control groups (cgroups)
        \item Union mount file system (aufs)
    \end{itemize}
  \end{frame}
  
  \section{Use cases}
  \begin{frame}
    
  \end{frame}
  
  \section{sbgsdk}
  \begin{frame}
    
  \end{frame}
  
\end{document}